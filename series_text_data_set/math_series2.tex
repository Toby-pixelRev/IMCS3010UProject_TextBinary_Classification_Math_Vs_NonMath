\usepackage{amsmath}

\begin{document}

\text{The difference between a Taylor polynomial and a Taylor series is the former is a polynomial,} \\ 
\text{containing only a finite number of terms, whereas the latter is a series, a summation of an infinite} \\ 
\text{set of terms. When creating the Taylor polynomial of degree} \ n \ \text{for a function} \ f(x) \ \text{at} \ x = c \ \text{we needed} \\ 
\text{to evaluate} \ f\text{, and the first} \ n \ \text{derivatives of} \ f \ \text{, at} \ x = c. \ \text{When creating the Taylor series of f, it helps} \\ 
\text{to find a pattern that describes the} \ n^{th} \ \text{derivative of} \ f \ \text{at} x = c. \ \text{We demonstrate this in the next two examples.} \\

\end{document}