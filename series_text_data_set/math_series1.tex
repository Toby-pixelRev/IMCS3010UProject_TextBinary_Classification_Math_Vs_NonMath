\usepackage{amsmath}

\begin{document}

\text{The difference between a Taylor polynomial and a Taylor series is the former is a polynomial, containing only a finite number of terms, whereas} \\ 
\text{the latter is a series, a summation of an infinite set of terms. When creating the Taylor polynomial of degree n for a function} \ \begin{equation}f(x)\end{equation} \ \text{at} \ \begin{equation}x = c\end{equation}, \\
\text{we needed to evaluate} \ \begin{equation}f\end{equation}, \text{and the first} \ \begin{equation}n\end{equation} \ \text{derivatives of} \ \begin{equation}f\end{equation}, \ \text{at} \ \begin{equation}x = c\end{equation}. \\ 
\text{When creating the Taylor series of} \ \begin{equation}f\end{equation}, \ \text{it helps to find a pattern that describes the} \ \begin{equation}n^{th}\end{equation} \ \text{derivative of} \ \begin{equation}f\end{equation} \ \text{at} \ \begin{equation}x = c\end{equation}. \\
\text{Let} \ \begin{equation}f(x)\end{equation} \ \text{have derivatives of all of orders at} \ \begin{equation}x = c\end{equation}. \\ 
\text{1. The Taylor Series of} \ \begin{equation}f(x)\end{equation}, \ \text{centered at} \ \begin{equation}c\end{equation} \ \text{is} \ \begin{equation}\sum^{\infty}_{n=1} \frac{f^{(n)}(c)}{n!}(x-c)^{n}\end{equation} \\
\text{2. Setting} \ \begin{equation}c = 0\end{equation} \ \text{gives the Maclaurin Series of} \ \begin{equation}f(x)\end{equation}: \ \begin{equation}\sum^{\infty}_{n=1} \frac{f^{(n)}(0)}{n!}x^{n}\end{equation} \\
\end{document}


